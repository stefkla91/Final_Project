\documentclass[10pt,a4paper]{article}
\usepackage[utf8]{inputenc}
\usepackage{amsmath}
\usepackage{amsfonts}
\usepackage{amssymb}
\usepackage{graphicx}
\usepackage{cite}
\usepackage{ucs}
\usepackage{listings} %for code
\usepackage{color} %for code
\usepackage{ulem} %dobule underline
\usepackage[english]{babel}
\usepackage[margin=2.5cm]{geometry}
\usepackage[hidelinks]{hyperref}

\author{Stefan Klaus\\stk4@aber.ac.uk\\Aberystwyth, Wales}
\title{Major Project Documentation}

% defines how code will be shown
\definecolor{dkgreen}{rgb}{0,0.6,0}
\definecolor{gray}{rgb}{0.5,0.5,0.5}
\definecolor{mauve}{rgb}{0.58,0,0.82}
% still code
\lstset{frame=tb,
  language=Java,
  aboveskip=3mm,
  belowskip=3mm,
  showstringspaces=false,
  columns=flexible,
  basicstyle={\small\ttfamily},
  numbers=none,
  numberstyle=\tiny\color{gray},
  keywordstyle=\color{blue},
  commentstyle=\color{dkgreen},
  stringstyle=\color{mauve},
  breaklines=true,
  breakatwhitespace=true
  tabsize=3
}

%%%%%%%%%%%%%%%%%%%%%%%%%%%%%%%%%%%%%
%	usefull stuff:
%	[3ex] is better than vspace{11pt}
%	\uuline{} gives a double underline
%	
%	\begin{figure}[h]
%	\includegraphics[width = 1\textwidth]{path_to_file}
%	\caption{caption_for_the_picture}
%	\label{Figure 1}
%	\end{figure}
%	
%	centering a small picture
%	\centerline{\includegraphics[width=0.9\textwidth]{file}}
%	
%	start code listing:
%	\begin{lstlisting}
%	\end{lstlisting}
%%%%%%%%%%%%%%%%%%%%%%%%%%%%%%%%%%%%
\begin{document}
\bibliographystyle{plain}
\maketitle
\newpage
\tableofcontents
\newpage
\begin{flushleft}
\section{Background and Design}
\subsection{What the Project is about}
This project is about the map building using swarm intelligence.\\
The aspects of the project include SLAM(Simultaneous Localization And Mapping) as well as communication between a swarm of robots.\\
This document will disuse aspects such as robot localisation, deployment of the swarm, communication between robots and environment mapping.\\
My work is based on the research themes in these areas and I am using the robot simulator software Webots\textsuperscript{\texttrademark}. The mobile robot platform I am using is a virtual representation of the E-Puck robot. \\

\subsection{Background}
\subsubsection{SLAM - Simultaneous Localization And Mapping}
The SLAM problem is a current research topic which is based on different localisation algorithms and using a range of different sensor to effectively map an target area. A lot of different approaches have been done and many research papers have been written, the one I am basing my project on being a paper about a SLAM solution designed for autonomous vehicles\cite{Dissanayake2001Solution}.\\
While the research area of this paper is based on a much larger scale and only using 1 vehicle in comparison to flock, it does still give me an insight upon the SLAM problem.\\
E.g. the problem with localisation in an dynamic environment, the paper tackles this problem by using global reference points and a millimetre wave radar, however for my project I do use an static environment and simple laser range finders. So this paper is only used a reference to the localisation problem, especially the idea of using "global" reference points for the created map.\\
As the test environment and the sensors available for the E-puck sensors are limited I will assume that the starting location of the robots is known.\\[3ex]

Another paper I read about this problem used an approach much more similar to my own project, by using different mobile robots which have no GPS access and simply use 2D laser range finders. However the approach described in this paper was based around the mapping of one "lead" robot and the traversing the same map again with a second robot using the map generated by the first for localisation purposes.\\
The second robot would then scan the target area again and refine the already generated map though using the (now stationary) first robot as an reference point. Since my project is planed about using multiple robot which scan the area at the same time it does still gives some information and idea about map sharing and improving. \\
Since my swarm will traverse the area at the same time it would be needed to share the map in real time and know the robot location in comparison to the first global referring point.  By implementing this I could rescan an area if a robot traverse an area another robot already scanned and refine the the final map by this. I will however look into real time sharing between different robots and can not say at this point if I will implement this in my final solution, it is however a interesting thought.\\[3ex]

\subsubsection{Deployment}
The deployment strategy is an important part of my project as it defines how effective the swarm will cover the target area which will define how long it will take to scan and map the whole area. Another important aspect is how many robots can the swarm hold and effectively deploy using the current deployment strategy. \\ 
One research paper I found proposed an solution of a communication network where the comm nodes keep track of the robots positions and guide them in directions which have not been explored in the last time period\cite{Batalin2003Coverage}. The paper uses a solution which is based on small comm nodes deployed by the robot, to make the solution fitting for my project I would have to define some of my E-pucks as communication nodes which remain on a fast position and guide the "scout" E-pucks based on area which have been least visited by the other robots.\\ 
This is certainly an possible solution however it could be considered a waste of robots in an small environment. Since I am using an simulator there is no communication range problem, but as my indentions are to make it as close to a possible real world application as possible I must still consider this.
That is why I am going to implement an maximum communication range for the robots however more about that can be found inside the "Communication" section. It is however an aspect to which I will come back later in my testing.\\[3ex]

Another paper proposed an solution which is based on an Nearest-Neighbour algorithm, meaning every robot must have always a minimum of "N" other robots inside its communication range\cite{Poduri2004Constrained}. In this solution to robots would emit signals to other robots which would manoeuvre the robots away from each other until only "N" robots remain inside the robots communication range. \\
This solution would cover a large area with the swarm fairly fast and also be adaptable for a swarm of any size, however a solution of moving the whole swarm in on direction would be needed in order to cover the whole target area, assuming the robot swarm is not big enough to cover it once completely deployed.
This would therefore need both a lead robot which decides the movement of the swarm and communication robots which would always have at least 1 link to another comm robot in order to have a communication line back to the lead robot. \\
This is one of two 2 deployment strategies I will try to implement during the testing period and try to find out which one would be more fitting for my project.\\[3ex]

\nocite{*}
\newpage
\bibliography{stk4_major_project}
\end{flushleft}
\end{document}

