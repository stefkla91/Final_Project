\chapter{Evaluation}
\label{Evaluation}
The overall result of this project failed to reach the goal set in the beginning of the project, to map an environment which holds obstacles. However the program does achieve a very good results inside the environment it was developed in. I think the main reason the project failed to reach the set goal is based on a combination of being to ambitious and errors in the design.\\[3ex]

The reasons I believe this are based around the fact that I did not completely understand the time requirements needed in order to implemented the program and circumvent the problems sole odometry based movement and localisation calculations hold. \\
During this project I have spend a lot of time improving the deployment pattern, the control loop and the different algorithms I implemented to improve the localisation and movement. 
However I do believe that the results which have been reached clearly show a vast improvement of what my program can do compared to the time of the mid-project demo. \\
Since then the movement pattern were implemented, as well as all other algorithms which have been described in chapter 3. \\
Still the aim set for this project has not been entirely reached.
It is very close and it would not take too much work to fix the problems discovered in the testing period for a future implementation. \\[3ex]

I do think that the requirements for this project have been correctly identified at the time, as well as \textit{most} of the design choices. The only choices which I think now should have been different at the time was the choices of movement pattern.\\
While I do not think that this movement is a completely wrong approach, it should have been designed different. 
Especially I am thinking about what changes could have been done to the behaviour of reaching an obstacle. 
The program at the moment does simply proceed with its U-Turn pattern which, as discussed in chapter 4, causes a number of problems with mapping the entire outline of obstacles. 
While the implemented movement pattern does cover the entirety of an empty room(obviously as chapter 4 showed there are a number of bugs which would have to be fixed) it should have been considered with the mapping of obstacles in mind, maybe a wall following approach which tracks and maps the outline of an obstacle after it has been encountered would have been more appropriate.\\
While I do not think that implementing such functionality in the future is a problem, the results might have been different it it would have been design with this feature in mind.\\[3ex]

I think that the set of tools which I have chosen for this project, the Webots\textsuperscript{\texttrademark} simulator and the C language, have been a good choices. 
The Webots\textsuperscript{\texttrademark} API provided all the functionality which was needed for the implementation of the project,  the only problem encountered with Webots\textsuperscript{\texttrademark} simulator was the fact that no exact measurement tools to measure the robots movement are available. \\[3ex]

All in all, even though the aims of the project were not completely reached, I am content with what has been achieved. 
The aim of mapping a room with obstacles accurately has not been reached, but the results achieved in an empty room showed of what the created program is capable. 
If more time were available more functionality could be added and the existing functions further calibrated in order to fix the problems encountered by adding obstacles to the test environment. \\
I do not think that bad time management was a problem in this project, it was rather being to ambitious in the beginning of the project without knowing the problems which can be encountered by "limiting" my self to only using odometry calculations. \\[3ex]

I have learned a lot about odometry during this project and I am sure that this would help greatly in a future implementation of this or a similar project. \\
I have learned what needs to be considered in the planning phase of a project and I am sure that a future project which consists of the deployment of wheeled robots I would manage a better design which considers the limitations of odometry and deployment techniques from the get-go.




