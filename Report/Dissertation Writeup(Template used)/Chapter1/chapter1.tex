\chapter{Background \& Objectives}
\label{Background}
\section{Aim of the project}
\label{aim}
The aim of this project is to map an environment using a single robot. \\
The key aspects of this project is to not use GPS or compass sensors and rather calculate the position and orientation of the robot using only its sensors and odometry calculation. \\
The inspiration for this project is taken from the idea of mapping inside areas where GPS or compass information are not always reliable enough to be used. \\[3ex]

The task of mapping an empty room has been achieved, but in future changes need to be made to the program in order to map rooms with obstacles.\\
In the following document the implementation and testing results are discussed in further depth.

\section{Information about the project}
\subsection{Choice of development environment and programming language}
Early in the year when the outline of the project was planned it was considered if a simulator should be used or rather a real robot.\\
After consolation with the project supervisor the decision fell to use a simulator as it allowed for easy changes in the environment as well as other reasons. Such reasons were the much more apparent ease with which the robot could be located inside the environment as the approach of using only odometry calculations to located the robot has been deemed ambitious. Another reason was the ease with which multi robot usage and other changes could be implemented if enough time were available at the end of the project.\\
It was decided to use the Webots\textsuperscript{\texttrademark} simulator as it provided a number of good modelling tools for the environment as well as already having the E-Puck robot platform integrated. The simulator also generates noise in the environment in form of friction between robot wheels and the floor as well as noise inside distance sensor readings, which made for a more challenging and interesting project.\\
The programming language which this project will be developed in is C. The reason C has been chosen is the fact that a prior knowledge of C exists and the Webots\textsuperscript{\texttrademark}  C++ API is compatible with the C language.

\subsection{Choice of robot}
For this project it has been decided to use the E-Puck robot platform. \\
Reason for this is the range of sensors available for this platform. The E-Puck robot comes with 8 distance sensors mounted around the top, with more sensors facing to the front than to the back. This is important as this allows for finding obstacles to the front and side to the robot while movement.
Also the E-Puck has only 2 drive wheels, which makes the odometry calculations easier than having to do work with 4 or more motor encoders. The main concern for the number of motor encoders was the time frame available for this project, and it was decided that using a 2 wheeled robot was sufficient as the environment is a clean indoor environment with even ground. And so using a 2 wheeled robot was deemed more accessible then increasing the calculation difficulty with more motors. Another bonus was that a virtual representation of the E-Puck robot has already been integrated inside the Webots\textsuperscript{\texttrademark} simulator.\\

\subsection{Choice of environment design}
It was decided to use a simple environment design in which to implement and test the project.\\
The reasons for that being that it would be easier to map environment with distinctive outlines and obstacles, as the movement and localisation was considered to be the largest challenge of the project.
The decision fell to use empty environments while implementing the localisation and movement algorithms as results from these environment as quit clear to interpret, while obstacles could cause problems. However the aim of the project was not to map only empty rooms, so obstacles would be added to the environment as soon as the movement and localisation algorithms were in place. 
More information about the environment design can be found in chapter 2. 

\section{SLAM - Simultaneous Localization And Mapping}
The SLAM problem is a current research topic which is based on different localisation algorithms and using a range of different sensor to effectively map an target area. \\
The SLAM problems ask if it is possible for an autonomous vehicle to start in an unknown location in an unknown environment and then to incrementally build a map of this environment while simultaneously using this map to compute absolute vehicle location\cite{Dissanayake2001Solution}.
A lot of different approaches have been done and many research papers have been written, the one this project is based on is a paper about a SLAM solution designed for autonomous vehicles\cite{Dissanayake2001Solution}.\\
While the research area of this paper is based on a much larger scale, it does still give me an insight upon the SLAM problem.\\
E.g. the problem with localisation in an dynamic environment, the paper tackles this problem by using global reference points and a millimetre wave radar. For this project an static environment and simple laser range finders are used. So this paper is only used a reference to the localisation problem, especially the idea of using "global" reference points for the created map.\\
As the test environment and the sensors available for the E-puck sensors are limited this project will assume that the starting location of the robots is known.\\[3ex]

Another paper which was read about this problem used an approach much more similar to this project, by using different mobile robots which have no GPS access and simply use 2D laser range finders. However the approach described in this paper was based around the mapping of one "lead" robot and the traversing the same map again with a second robot using the map generated by the first for localisation purposes.\\
The second robot would then scan the target area again and refine the already generated map though using the (now stationary) first robot as an reference point. Since this project is using single a robot for mapping purposes and multi robot usage would only be added if enough time is available, this paper was not inherently useful, however gave some useful insight for information sharing between robots or sensor stations as well as localisation of robots using a global reference point.\\
Since this project aims at real time localisation and mapping communication with a "uplink" point would be essential to achieve this in a real world setting. By using a similar implementation to the one described in the paper it would be possible to rescan a mapped area if it is traversed again, and by that refine the mapping. This is requires however very good localisation techniques.


\section{Deployment}
The deployment strategy is an important part of this project as it defines how effective the robot will cover the target area which will define how long it will take to scan and map the whole area. 
One research paper which was read proposed an solution of a communication network where the comm nodes keep track of the robots positions and guide them in directions which have not been explored in the last time period\cite{Batalin2003Coverage}. The paper uses a solution which is based on small comm nodes deployed by the robot, to make the solution fitting for this project the developer would have to use multiple E-Pucks and define some of these as communication nodes which remain on a fast position and guide the "scout" E-puck based on area which have been least visited by the other robots.\\ 
However since multi-robot usage is planned as an addition if enough time is available this deployment strategy is not fitting for the major part of work.\\[3ex]

Another ,more suitable, deployment algorithm is shown on the paper \textit{"Deployment of Mobile Robots With Energy and Timing Constraints"} by Yongguo Mei \textit{at al.}\cite{Mei2006Deployment}. In this paper a controlled movement pattern of different \textit{sectors} is proposed. The paper is centred on the exploration with multiple robots. Since this project will only use a single robot, this paper is not entirely applicable however part of the deployment strategy is. In the paper the robots deploy in an \textit{scanline pattern}, a rectangular movement pattern which allows the robot to scan obstacle to the sides of the movement route. \\
Figure \ref{Figure 3} on page \pageref{Figure 3} shows the movement pattern after being changed to single robot usage.\\
This seems very effective in terms of complete overall environment coverage as the environment design will be suitable for single robot exploration.

\section{Analysis}
The background reading clearly showed the main aspects which were needed.\\
These were an effective deployment pattern, localisation of the robot and sensing of obstacles to map. 

\subsection{Deployment pattern}
An effective deployment pattern is needed as it is important to traverse the whole of the environment in order to map all obstacles inside it.\\
Characteristics for a effective deployment pattern are to traverse the whole environment and doing so in as short a time frame as possible. Other considerations for the deployment pattern are to be easily traceable that is, to be able to understand the choices the robot makes and thereby be able to improve the deployment algorithm during the implementation phase. For this reason a controlled movement pattern has been implemented inside the program, more information about said pattern and which other patterns were considered can be found in chapter 2. \\
To be able to use the controlled movement pattern a number of functions need to be created. 
These functions include to be able to move an accurate distance and to be able to turn a given number of degrees, as this is needed for controlled movement of the robot. 

\subsection{Localisation}
Arguably the most important aspect of the project is to have a good localisation solution, since without the mapping will be ineffective. There are a number of different approaches which can be done for this, papers which were researched showed approaches such as using global reference points\cite{Dissanayake2001Solution, Thrun2000Realtime}, GPS and compass data as well as approaches which do not require data from sensors like GPS or compass modules\cite{Poduri2004Constrained}.\\
One of the aspects wanted for this project is to be able to locate the robot without the use of GPS or compass data and rather use pure odometry calculations to reach a solution of the localisation problem.\\[3ex]

To be able to use odometry calculations the starting position of the robot needs to be known so that the positions changes relative to the starting point can be calculated. Encoder steps of the robots stepper motors will be used to calculate the movement of the robot and the turning. \\
To know accurately how far the robot has moved and how many degrees its turned is important in order to be create concise mapping. \\
For this aspect functions need to be created which report the accurate movement so that the new position can be calculated. This goes hand in hand with the movement functions created for the deployment algorithm, as they are required to move and turn the robot with high accuracy, the same functions can be used to report the robots new encoder positions to the functions calculating the position. 

\subsection{Mapping}
To be able to create a map of the environment the right sensing method needs to be found, i.e. different sensors perform different in similar environments, also the range of sensors varies a lot. 
The choice of sensors is limited as the work is based on the E-Puck robot platform.
However the quality of the mapping is strongly dependent on the deployment pattern and the localisation solution.\\

The sensors used on the virtual E-Puck are standard laser range finders which are used to locate obstacles map them. As already mentioned the accuracy of the mapping depends majorly on the accuracy of the robots movement and localisation algorithms. 

\section{Process}
The life cycle model used for this project is "Feature driven development", as it seems more appropriate for this project as other models. \\
The reason FDD was chosen was as this is a single person project , and the requirements allowed for easy distinguish between features which need to be developed. \\
For example: the movement pattern needs to be in place before the localisation can be implemented. When the localisation is implemented the mapping can be implemented as the features build on each other. \\

A few changes were made to it. Such as all tests are done only on one environment during the development process. The milestones for the different features were rather small and have been incrementally changed over time. E.g: the milestones for the movement changed from "move a given distance" to "move through the entire environment" in the time duration of the project.


